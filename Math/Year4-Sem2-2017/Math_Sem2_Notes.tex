\documentclass[10pt,a4paper]{article}
\usepackage[utf8]{inputenc}
\usepackage{amsmath}
\usepackage{amsfonts}
\usepackage{amssymb}
\setlength{\voffset}{-0.75in}
\setlength{\textheight}{700px}
\begin{document}
\title{
Mathematics \\
\large Semester 2 2017
}
\author{Cxo05}

\maketitle

\section{Methods of Proof}
Proofing stuff, trivial right? Probably not. Presentation plays a large part on whether you get the full mark or not.

\subsection{Definitions}
\subsubsection{Statements}
A statement is a sentence that is true or false but not both.
\subsubsection{Conditional statements}
Conditional statements are statements where the truth of a statement is conditioned on the truth of another statement.
\begin{center}
$p \rightarrow q$
\end{center}
The above denotes that "If $p$, then $q$". This statement is false when $p$ is true and $q$ is false; otherwise it is true.
\newline\newline
\textbf{Example} 
If $n$ is an odd integer, prove that $n^2$ is an odd integer.
\newline


\end{document}