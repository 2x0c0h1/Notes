\documentclass[10pt,a4paper]{article}
\usepackage[utf8]{inputenc}
\usepackage{amsmath}
\usepackage{amsfonts}
\usepackage{amssymb}
\usepackage{romannum}
\setlength{\voffset}{-0.75in}
\setlength{\textheight}{700px}
\begin{document}
\title{
Mathematics \\
\large Semester 2 2017
}
\author{Cxo05}

\maketitle
\section{Methods of Proof}
Proving stuff, trivial right? Probably not. Presentation plays a large role in determining your score.

\subsection{Definitions}
\subsubsection{Statements}
A statement is a sentence that is true or false but not both.
\subsubsection{Conditional statements}
Conditional statements are statements where the truth of a statement is conditioned on the truth of another statement.
\begin{center}
$p \rightarrow q$
\end{center}
The above denotes the statement ``If $p$, then $q$'' which can also be written as ``$p$ implies $q$''. This statement is false when $p$ is true and $q$ is false; otherwise it is true.
\subsubsection{Converse of a statement}
The converse of $p \rightarrow q$ is $q
\rightarrow p$ . These statements are \textbf{NOT} logically equivalent. 
\subsubsection{Contrapositive of a statement}
The contrapositive of $p \rightarrow q$ is $\sim q \rightarrow \sim p$ . These statements are logically equivalent, thus the original statement is true when its contrapositive is true.
\subsubsection{Inverse of a statement}
The inverse of $p \rightarrow q$ is $\sim p \rightarrow \sim q$ . These statements are \textbf{NOT} logically equivalent. 
\subsection{Proof by contradiction}
\begin{enumerate}
  \item Suppose that the statement to be proved is false.
  \item Show that the assumption leads to a contradiction.
\end{enumerate}
Since a state cannot be both true and false, the statement being false leads to a contradiction, therefore it must be true. 
\begin{center}
$r : p \rightarrow q$
\end{center}
\begin{center}
$\sim r : \sim (p \rightarrow q)$
\end{center}
\begin{center}
$\sim r : p \wedge \sim q$
\end{center}
\subsection{Proof by mathematical induction}
\begin{enumerate}
  \item Let $P_n$ be the statement you want to prove. Add the range of n as well.
  \item Show that $P_n$ is true for the base case of $n$.
  \item Let $k$ be an arbitrary integer such that its range is the same as the range of n and suppose that the statement $P_k$ is true.
  \item Expand the LHS of $P_{k+1}$ to include our assumption of $P_k$.
  \item Continue doing math until you reach the RHS of $P_{k+1}$
  \item Thus, $P_k \rightarrow P_{k+1}$ 
  \item Hence, by mathematical induction, $P_n$ is true for all integers n, within its specified range.  
\end{enumerate}

\section{Differentiation \Romannum{3}}
Revisit


\subsection{Local Linear Approximation}
Lets say you have a function $f$ and you know a point on this particular function. We can use Local Linear Approximation to find an estimate of the function at values close to the point we know.
\\\\
Find the tangent to curve $y = f(x)$ at $P(a, f(a))$, $f(a)$ being a result we know or can easily get. It can be done by differentiating the function. We can thus get the following estimate.
\begin{center}
$f(x_1)\approx f(a)+f'(a)(x_1-a)$
\end{center}   
\subsection{Absolute (Global) Extrema}
A function $f$ has a absolute maximum or minimum at $c$ if $f(c)\geq f(x)$ or $f(c)\leq f(x)$ for all $x$ in the domain of $f$ respectively.
\subsubsection{Critical points}
A point $(x, f(x))$ is a critical point if either $f'(x)=0$ or $f'(x)$ does not exist. A critical point cannot be an endpoint of the domain.
\\\\
To find the absolute max or min, we first find the critical points of the function and its evaluate its endpoints. The absolute max is the maximum value of the critical points and endpoints. Vice versa.  
\subsection{Increasing and decreasing functions}
With increasing values of $x$, the function $f(x)$ increases. Vice versa.
\\\\
If $f'(x)>0$ for all $
\vfill
\fbox{
This thus concludes the summary for Year 4 Mathematics Semester 2 2017.
}
\end{document}