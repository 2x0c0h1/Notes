\documentclass[10pt,a4paper]{article}
\usepackage[utf8]{inputenc}
\usepackage{amsmath}
\usepackage{amsfonts}
\usepackage{amssymb}
\usepackage{romannum}
\setlength{\voffset}{-0.75in}
\setlength{\textheight}{700px}
\begin{document}
\title{
Mathematics \\
\large Semester 2 2017
}
\author{Cxo05}

\maketitle
\section{Methods of Proof}
Proving stuff, trivial right? Probably not. Presentation plays a large part on whether you get the full mark or not.

\subsection{Definitions}
\subsubsection{Statements}
A statement is a sentence that is true or false but not both.
\subsubsection{Conditional statements}
Conditional statements are statements where the truth of a statement is conditioned on the truth of another statement.
\begin{center}
$p \rightarrow q$
\end{center}
The above denotes the statement ``If $p$, then $q$''. This statement is false when $p$ is true and $q$ is false; otherwise it is true.
\subsubsection{Converse of a statement}
The converse of $p \rightarrow q$ is $q
\rightarrow p$ . These statements are \textbf{NOT} logically equivalent. 
\subsubsection{Contrapositive of a statement}
The contrapositive of $p \rightarrow q$ is $\sim q \rightarrow \sim p$ . These statements are logically equivalent, thus the original statement is true when its contrapositive is true.
\subsubsection{Inverse of a statement}
The inverse of $p \rightarrow q$ is $\sim p \rightarrow \sim q$ . These statements are \textbf{NOT} logically equivalent. 
\subsection{Proof by contradiction}
\begin{enumerate}
  \item Suppose that the statement to be proved is false.
  \item Show that the assumption leads to a contradiction.
\end{enumerate}
Since a state cannot be both true and false, the statement being false leads to a contradiction, therefore it must be true. 
\begin{center}
$r : p \rightarrow q$
\end{center}
\begin{center}
$\sim r : \sim (p \rightarrow q)$
\end{center}
\begin{center}
$\sim r : p \wedge \sim q$
\end{center}
\subsection{Proof by mathematical induction}
\begin{enumerate}
  \item Let $P_n$ be the statement you want to prove. Add the range of n as well.
  \item Show that $P_n$ is true for the base case of $n$.
  \item Let $k$ be an arbitrary integer such that its range is the same as the range of n and suppose that the statement $P_k$ is true.
  \item Expand the LHS of $P_{k+1}$ to include our assumption of $P_k$.
  \item Continue doing math until you reach the RHS of $P_{k+1}$
  \item Thus, $P_k \rightarrow P_{k+1}$ 
  \item Hence, by mathematical induction, $P_n$ is true for all integers n, within its specified range.  
\end{enumerate}

\section{Differentiation \Romannum{3}}
\vfill
\fbox{
This thus concludes the summary for Year 4 Mathematics Semester 2 2017.
}
\end{document}