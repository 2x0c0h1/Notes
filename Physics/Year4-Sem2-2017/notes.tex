\documentclass[a4paper]{article}
\usepackage[utf8]{inputenc}
\usepackage{amsmath}
\usepackage{amsfonts}
\usepackage{amssymb}
\usepackage{graphicx}
\usepackage{flexisym}
\usepackage{geometry}
\setlength{\voffset}{-0.75in}
\setlength{\textheight}{700px}
\begin{document}

\title{
Physics \\
\large Semester 2 2017
}
\author{Cxo05}

\maketitle

\section{Electric Charges and Fields}
\subsection{Developing a Charge model}
\begin{itemize}
\item Frictional forces can add for remove charges from an object (Charging)
\item Two like charges exert repulsive forces on each other. Opposite charges attract each other
\item Force between two charges is a long range force
\item Charge can be transferred from one object to another, but only when in contact. Removing charge (Discharging) can be done by touching the object.
\end{itemize}
\subsection{Charge}
There are two types of charges, positive (+ve) and negative (-ve) charges.
It is convention that a glass rod when rubbed with silk is +ve. Anything that attracts the glass rod is -ve, vice versa. Electrons are -ve and protons +ve. 
\subsection{Insulators and conductors}
Conductors are materials through or along which charge easily moves. Insulators are materials on or in which charges remain fixed in place. \textbf{BOTH} can be charged.
\subsection{Charging and discharging}
Insulators are often charged by rubbing. Metals cannot be charged by rubbing due to it having mobile charges. A metal object can be charged by coming into contact with a charged plastic rod. Once charge is transferred to the metal and rod is removed, charges spread over the surface of the metal, attaining electrostatic equilibrium, where the charges are at rest. \\\\
An object can be discharged when a larger conductor touches the object. Any excess charge that was initially on the object can now spread over the larger conductor, becoming discharged. 
\subsection{Charge polarisation}
 




\end{document}